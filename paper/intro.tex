\section{Introduction}
\label{sec:intro}

Example of incorporation citations~\cite{coral:nsdi04}.

Man in the middle (MITM) attacks pose great risk to web users. An attacker can serve malicious web content to users and act on their behalf. The canonical way of dealing with MITM attacks is through encryption: using SSL to make sure communication is protected. In this work, we aim to provide a MITM detection mechanism for users that access websites that do not support SSL. Such lack of support may be due to limited resources, neglect or security-unsavvy web administrators. Our project provides protection for users from becoming victims of hijacked HTTP traffic, without requiring support from the admin of the website. The control is given to the end user in whether she would like to surf the web in a more secure way.

We would like to achieve the following goals:
A system that detects whether HTTP web content was changed en route.
Lightweight compared to VPN.
Can be deployed without the cooperation of the site owner.

Threat Model:
The attacker can only control a client’s access link, and is distant enough from the server’s link to not control all server-to-client traffic. We assume the client is the target (for example, attacker uses an unsecure WiFi network which he shares with client).
An attack in which the adversary has compromised the server’s link is out of the scope of this work.
